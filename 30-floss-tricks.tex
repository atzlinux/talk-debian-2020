% 30-floss-tricks.tex

\begin{frame}{开源社区协作技巧}
	\begin{itemize}
		\item 除非是完全国人开发维护的项目,否则请全程使用英文。
		\begin{itemize}
		    \item 社区通用的协作语言是英语,使用中文意味着别人只能用谷歌翻译理解你的内容,某些上游可能初看中文看不懂就直接无视你的贡献。
		    \item 请尽量使用正确英文语法,避免使用缩写(除非缩写已约定俗成)。
		    \item 请只在全中文项目中使用中文:判断方法很简单,如果大多数其它讨论都是中文,那么可以使用中文。
		\end{itemize}
	    \item 提Pull Request或提供补丁时尽量原子化、简单化。
	    \begin{itemize}
	    	\item 一个补丁精准解决一个问题,避免无关修改(如不必要的空格/制表符/换行)
	    	\item 提交说明中准确描述补丁的作用
	    	\item 使用Git协作时善用\texttt{git rebase},即使开发时有多个提交,在提出PR时要基于上游最新代码变基,并将自己的提交合并至单一提交或少数几个意义清晰的提交
	    \end{itemize}
        \item 讨论时永远不要吵起来
        \begin{itemize}
        	\item 任何情况下都对你没有任何好处
        \end{itemize}
	\end{itemize}	
\end{frame}

\begin{frame}{开源社区协作技巧(二)}
	\begin{itemize}
		\item “固定”你的身份
		\begin{itemize}
			\item 与他人交流时选择一个全名(或昵称),选择一个(半永久)电子邮箱地址,不要常常改动
			\item 有助于其他人认识你,方便建立身份和声望(“Reputation”)
		\end{itemize}
		\item 讨论时给出充足的引用信息
		\begin{itemize}
			\item 如需引用某段文档,应同时给出对应网页超链接
			\item 考虑类似科技文献的引用方式(\texttt{[1][2]...})
			\item 在支持Markdown等标记语言的场景下(如GitHub Issue)更加方便(Markdown等标记语言原生支持超链接等语法)
		\end{itemize}
		\item 基于电子邮件的协作
		\begin{itemize}
			\item 如果可能,请用纯文本电子邮件(非HTML电子邮件)
			\item 回复讨论时适当引用原文
			\item 遵循英文信件书写格式(或者看看其他人的写法,模仿来进行书写)
			\item 如果常常用电子邮件打交道,建议选择一个靠谱的电子邮件客户端(如Evolution、Thunderbird等)
			\item 保持耐心和礼貌(邮件不一定会得到及时回复),可以一周一封提醒邮件,做好持续数周/数月的准备
		\end{itemize}

	\end{itemize}	
\end{frame}