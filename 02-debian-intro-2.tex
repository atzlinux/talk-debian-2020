\begin{frame}{Debian的社区文化和特点}
       \begin{itemize}
               \item 对法律上的问题十分严肃,这尤其体现在对软件包的许可证审查上(通常会精细到源代码文件为单位进行审查);
               \item 开发流程上比较严谨\footnote{换句话说有官僚主义的倾向。};
               \item 软件包维护传统上有较强的“维护者强所有关系(strong package ownership)”的倾向,但近些年有将软件包由多个维护者共同维护或团队维护的趋势;这个特点对下游发行版向Deb
ian反哺贡献的效率有部分影响。
               \item 以志愿者模式参与开发和维护:Debian不是公司,大部分成员都是兼职维护Debian软件包/维护基础设施,少部分成员由其他公司雇佣并将以志愿者身份参与Debian开发作为工作一部
分(如Debian中存在Canonical公司\footnote{Canonical开发并维护Ubuntu,Ubuntu是Debian的衍生发行版。}员工等)。
       \end{itemize}
\end{frame}

\begin{frame}{Debian的社区文化和特点(二)}
       \begin{itemize}
               \item 主要以协商和讨论的方式决定开发方向;技术问题争论无果时会由数位资深开发者组成的技术委员会(“CTTE”,Technical Committee)作出决断,重大技术/非技术问题会由全体投票的方式决定方案\footnote{\url{https://www.debian.org/vote/}};
               \item 主要使用邮件列表、电子邮件和IRC进行交流;
               \item 在可能的情况下尽量给用户提供多种选择,软件功能在合理的前提下尽可能多地启用。例如:
               \begin{itemize}
                       \item 提供多种初始化(init)系统的选择;
                       \item 同时提供并维护多种桌面环境的软件套件;
                       \item 软件库的构建flag能启用的会尽量启用(如果发现有缺失的功能可以向维护者建议修复);
                       \item 然而某些部分严格遵循标准,如CPU Baseline(AMD64不会启用SSE3/SSE4/AVX)\footnote{但Debian为特殊的软件包提供了变通方式,详见 \href{https://tracker.debian
.org/pkg/isa-support}{isa-support} 源码包内容。};
               \end{itemize}
       \end{itemize}
\end{frame}