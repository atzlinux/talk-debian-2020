% 01-debian-intro.tex

\begin{frame}{Debian简介}
	\begin{itemize}
		\item 最古老经典的Linux\footnote{Debian同时有将Linux内核替换为FreeBSD内核的实验性版本。}发行版之一:始于1993年(Linux v0.01于1991年发布);
		\item 成为了许多衍生发行版的基石:Ubuntu、Deepin、Linux Mint等;
		\item 承诺所包含软件全自由:严格遵循《Debian自由软件指导方针》\footnote{某些硬件的闭源驱动和固件、微码等使用非官方的\texttt{non-free}部分提供,默认安装不启用},所有源代码可下载可在线浏览\footnote{可以在 \href{https://sources.debian.org}{sources.debian.org} 浏览Debian全部\texttt{main}部分的软件包源代码。};
		\item 开发了强大的软件包管理工具:apt和dpkg;
		\item 完全由社区运营维护,发展方向由Debian正式开发者(Debian Developer)经讨论与投票决定,不受商业公司控制;\footnote{法律支持和款项管理由非营利性质的SPI基金会提供支撑。}
		\item 大约每两年发布一个稳定版,每个版本支持时间约为五年;下一个大版本会是2021年5月的Debian 11(Bullseye)。
	\end{itemize}
\end{frame}

\begin{frame}{Debian的社区文化和特点}
	\begin{itemize}
		\item 对法律上的问题十分严肃,这尤其体现在对软件包的许可证审查上(通常会精细到源代码文件为单位进行审查);
		\item 开发流程上比较严谨\footnote{换句话说有官僚主义的倾向。};
		\item 软件包维护传统上有较强的“维护者强所有关系(strong package ownership)”的倾向,但近些年有将软件包由多个维护者共同维护或团队维护的趋势;这个特点对下游发行版向Debian反哺贡献的效率有部分影响。
		\item 以志愿者模式参与开发和维护:Debian不是公司,大部分成员都是兼职维护Debian软件包/维护基础设施,少部分成员由其他公司雇佣并将以志愿者身份参与Debian开发作为工作一部分(如Debian中存在Canonical公司\footnote{Canonical开发并维护Ubuntu,Ubuntu是Debian的衍生发行版。}员工等)。
	\end{itemize}
\end{frame}

\begin{frame}{Debian的社区文化和特点(二)}
	\begin{itemize}
		\item 主要以协商和讨论的方式决定开发方向;技术问题争论无果时会由数位资深开发者组成的技术委员会(“CTTE”,Technical Committee)作出决断,重大技术/非技术问题会由全体投票的方式决定方案\footnote{\url{https://www.debian.org/vote/};
		\item 在可能的情况下尽量给用户提供多种选择,软件功能在合理的前提下尽可能多地启用。例如:
		\begin{itemize}
			\item 提供多种初始化(init)系统的选择;
			\item 同时提供并维护多种桌面环境的软件套件;
			\item 软件库的构建flag能启用的会尽量启用(如果发现有缺失的功能可以向维护者建议修复);
			\item 然而某些部分严格遵循标准,如CPU Baseline(AMD64不会启用SSE3/SSE4/AVX)\footnote{但Debian为特殊的软件包提供了变通方式,详见 \href{https://tracker.debian.org/pkg/isa-support}{isa-support} 源码包内容。};
		\end{itemize}
	\end{itemize}
\end{frame}