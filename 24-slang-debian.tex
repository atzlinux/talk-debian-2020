% 24-slang-debian.tex
\begin{frame}{Debian常用缩写}
	Debian社区语境下有很多不太通用但时常出现的英文缩写,初看可能无法理解其含义,但在社区内工作时需要有一定了解。
	\begin{description}
		\item[DD] 即“Debian Developer”,Debian社区正式成员,其PGP Key登记在案,有内部投票权、被选举权、可以无限制向Debian软件仓库上传已有软件的新版本、有内网机器SSH和LDAP帐号等;
		\item[DM] 即“Debian Maintainer”,其PGP Key登记在案,只能对某些特定的软件无限制地上传新版本(特定软件上传权限由其他Debian Developer授予),除此之外无任何权利;
		\item[Salsa] 即 \texttt{\href{https://salsa.debian.org/}{salsa.debian.org}} 站点,是Debian自建的GitLab CE实例;
		\item[ITP] 即“Intent To Package”,是打算打包某个软件的声明;
		\item[RFS] 即“Request For Sponsorship”,已经完成某个软件的打包需要某位Debian开发者协助审查并上传至官方仓库(Debian Archive,不是Debian所使用的GitLab仓库);
		
	\end{description}
\end{frame}